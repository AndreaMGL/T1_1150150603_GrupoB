\documentclass[a4paper,openright,12pt]{report}
\usepackage[spanish]{babel} 
\usepackage{multirow}
%usepackage[latin1]{inputenc}
\usepackage[utf8]{inputenc}
\usepackage{graphicx}
\usepackage{lmodern}
\usepackage{listings}
\usepackage{color}
\usepackage{fancyhdr}
\usepackage{hyperref}
\usepackage[left=2.5 cm,right=2cm,top=2.5 cm,bottom=3cm]{geometry}

\usepackage{Sweave}
\begin{document}


\pagestyle{fancy}
\renewcommand{\headrulewidth}{0pt}
\bigskip
\bigskip


{\setlength{\arrayrulewidth}{0.5mm}
\begin{tabular}{p{2 cm} | p{10
cm} p{3 cm}}
\multirow{3}{2cm}{\Large{\includegraphics[width=1.9 cm]{logtipo}}} 

& \Large{\textbf{Universidad Nacional de Loja}} \\

\end{tabular}
\bigskip
\bigskip

\begin{flushleft}


\raggedright{ 
\small{\textit{\textbf{Área de la Energía, las Industrias y los Recursos Naturales No Renovables}}}
}
\thinspace
\rule{1\textwidth}{0.04cm} 
\thinspace
\raggedleft{ 
\Large{\textsl{\\Carrera de Ingeniería en Sistemas}}}
\end{flushleft}
\bigskip
\bigskip
\bigskip
\bigskip
\bigskip

\begin{center}
\textbf{Examen }
\end{center}


\begin{flushleft}
\textit{\textbf{Autor:}}
\textit{Andrea Guayllas}
\begin{itemize}
\renewcommand{\labelitemi}{$\diamond$} 
\item \href{http://www.iralis.org/?q=node%2F10&paso=10&letra=G&id=7206}{AFINF7206}
\end{itemize}
\bigskip
\end{flushleft}
\thinspace
\bigskip


\begin{flushleft}
\textit{\textbf{Docente:}}
\textit{Ing: Pablo Ordo?ez}
\begin{itemize}
\renewcommand{\labelitemi}{$\diamond$} 
\item \href{http://www.iralis.org/?q=node%2F10&paso=10&letra=O&id=4796}{AFINF7206}
\end{itemize}
\bigskip
\bigskip
\end{flushleft}

\thinspace
\bigskip
\begin{flushleft}
\textit{\textbf{Fecha: }\today{}}
\end{flushleft}
\thinspace
\bigskip

\cfoot[Loja]{\small{\textsc{Loja-Ecuador\\ 2015}}}
\renewcommand{\footrulewidth}{0pt}

\newpage

\begin{center}\huge{}\textbf{Parte B}\end{center}
¿Es aplicable la ingeniería de software cuando se elaboran webapps? Si es así, ¿cómo puede modificarse para que asimile las características únicas de éstas?\\

Si es aplicable, ya que actualmente las webapps no son solo paginas informativas si no que han ido evolucionando y actualmente son aplicaciones que se han integrado con base de datos y se han convertido en piezas fundamentales de desarrollo. \\ \\

Un breve descripción del dataset Titanic\\ \\
Este dataset proporciona información sobre el destino de los pasajeros en el viaje fatal del trasatlántico Titanic, que se resumen de acuerdo a la situación económica (clase), el sexo, la edad y la supervivencia.\\

Uso\\
Titánico\\
Formato

Una matriz de 4-dimensiones resultante de tabular 2201 observaciones sobre 4 variables. Las variables y sus niveles son los siguientes:\\

No Nombre niveles \\
1  clas   1ra, 2da, 3era, Tripulación\\
2 Sexo     Hombre, Mujer\\
3 Edad     Niño,Adulto\\
4 sobrevivió No, Sí\\

Detalles\\

El hundimiento del Titanic es un evento famoso, y nuevos libros siguen siendo publicado sobre el tema. Muchos hechos-de conocidas las proporciones de los pasajeros de primera clase a la política de "mujeres y niños primero', y el hecho de que esa política no era un éxito completo en el ahorro de las mujeres y niños en la tercera clase se reflejan en la supervivencia tarifas de diversas clases de pasajeros.\\

Estos datos fueron recogidos originalmente por la Junta Británica de Comercio en su investigación del hundimiento. Tenga en cuenta que no hay un acuerdo completo entre las fuentes primarias como a las cifras exactas a bordo, rescatados, o perdidos.\\

Debido, en particular, a la película de gran éxito 'Titanic', los últimos años vieron un aumento en el interés público en el Titanic. Datos muy detallados sobre los pasajeros ya está disponible en Internet, en sitios como la Enciclopedia Titanica (http://www.rmplc.co.uk/eduweb/sites/phind).\\
\newpage
\Sconcordance{concordance:ExamenGrupoB.tex:ExamenGrupoB.Rnw:%
1 13 1 1 0 109 1 1 3 38 0 1 2 2 1 1 2 4 0 1 2 1 1}

Mostrar el dataset\\
\begin{Schunk}
\begin{Soutput}
, , Age = Child, Survived = No

      Sex
Class  Male Female
  1st     0      0
  2nd     0      0
  3rd    35     17
  Crew    0      0

, , Age = Adult, Survived = No

      Sex
Class  Male Female
  1st   118      4
  2nd   154     13
  3rd   387     89
  Crew  670      3

, , Age = Child, Survived = Yes

      Sex
Class  Male Female
  1st     5      1
  2nd    11     13
  3rd    13     14
  Crew    0      0

, , Age = Adult, Survived = Yes

      Sex
Class  Male Female
  1st    57    140
  2nd    14     80
  3rd    75     76
  Crew  192     20
\end{Soutput}
\end{Schunk}

¿Cuál es el número total de casos en el dataset?\\
El numero total de casos en el dataset es de:
\begin{Schunk}
\begin{Soutput}
[1] 2201
\end{Soutput}
\end{Schunk}

\end{document}
